Many existing works provide ad-hoc image analysis methods for specific plant
phenotyping scenarios, for example stem angle
estimation~\cite{das2017automated}, 3d plant shoot
reconstruction~\cite{scharr2017fast}, ... Generic geometric approaches can deal
with a large variety of plant structure~\cite{reeb2018quantification} but are restricted in terms of robustness to
occlusions.

Simulated rendering of plants for semantic segmentation have already
been developed, e.g.~\cite{duboudin_toward_2019} but to the authors' knowledge
they don't account for plant geometry in the learning process. This limits applications
to simple identification of scene objects relying on texture and basic shape only.

TODO: Blabla general 2D image segmentation.

TODO: Blabla general 3D segmentation.

%Instruments: RGB-D, Lidar, Laser, kinect -> RGB
%Segmentation method: 2D-3D
%Applications:?
% 
\alien{TODO: Shi Koostra vs. us:
large variety of plants, user adaptation, complexity of plants, works on real plants}{}
% see CVPPP2018 review of article
