\paragraph{Testing methods.} The test data is composed of four
distinct datasets:
\begin{enumerate}[A.]
    \item A simulated dataset similar to the training dataset (14 different individuals, 1008 images)
    \item A small annotated set of plants from real Arabidopsis thaliana (4 different individuals, 6 images);
    \item A larger set of real Arabidopsis thaliana for qualitative evaluation (12 different indivudals, 864 images);
    \item A set of real images from another plant (tomato) to evaluate the possibility of transfering the learning onto other species of plants (TODO).
\end{enumerate}
All datasets will be available there: TODO.

Evaluation is done both on the 2D segmentation and on the resulting segmented 3D point cloud. For datasets A and B,
we present a class by class quantitative evaluation of the 2D segmentation method. For dataset A, we additionally  provide
class by class quantitative evaluation of the 3D voxel segmentation, as well as a comparison of output point clouds
to the ground truth given by L-Py. For dataset C, we provide qualitative assessment of the results of the segmentation,
both in 2D and in 3D. For dataset D, we restrict ourselves to qualitative evaluation of the 2D segmentation.

The background class is not considered as a class in itself in the evaluation, but is presented in the qualitative example. It
is only used for contrast when reconstructing point cloud from voxel data.

\paragraph{2D Segmentation.}
Images from datasets A are segmented used the trained neural network. Then, the results of 2D segmentation are
compared to the ground truth provided by the blender simulation. Figure~\ref{fig:seg2d_res} presents a sample from
dataset A and its segmentation.

\begin{figure}[h!]
    \centering \includegraphics[width = 0.25\linewidth]{figures/00000_rgb.png}\quad
    \includegraphics[width = 0.25\linewidth]{figures/000_background.png}\quad
    \includegraphics[width = 0.25\linewidth]{figures/000_flower.png}

    \vspace{1em}

    \includegraphics[width = 0.25\linewidth]{figures/000_fruit.png}\quad
    \includegraphics[width = 0.25\linewidth]{figures/000_leaf.png}\quad
    \includegraphics[width = 0.25\linewidth]{figures/000_pedicel.png}
    \caption{Images segmented with our network. From left to right, then top to bottom: original image,
        background, flower, fruit, leaf, pedicel. White indicates a high score (1.0) for the class,
        black indicates a low score (0.0).} \label{fig:seg2d_res}
\end{figure}

\paragraph{3D Segmentation.}

\paragraph{Qualitative evaluation on real plants.}

\begin{figure}[h!]
    \centering \includegraphics[width = \linewidth]{figures/real_scans.png}
    \caption{Sample images from the dataset used for qualitative assessment of the method on real data} \label{fig:realscans}
\end{figure}

\begin{figure}[h!]
    \centering \includegraphics[width = \linewidth]{figures/3drec.png}
    \caption{3D reconstruction of real plants. Red: stem, green: leaf, white: pedicel, purple: fruit:, yellow: flower} \label{fig:rec3d}
\end{figure}

\paragraph{Transfer to other species.}


