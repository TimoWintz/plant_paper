Plant architecture is a complex geometrical object. Due to both genetic and
environmental variations, organs evolve in patterns which present great diversity
both between species and among individuals of the same species. Automated
reconstruction of plant structure from lab or in-field acquisitions remains a challenge in computer
vision. The most common method for complex measurements of plant structure
remains handmade measurements and image analysis is a major bottleneck in plant phenotyping
\cite{minervini2015image}. The application of automated processing of plant structure are many: precise
quantifying of biomass and yield for agricultural crops, estimation of environmental response
of crops~\cite{peirone2018assessing, srivastava2017drought},  mapping genotypes to
phenotypes, estimating growth parameters of species, among other~\cite{denis1998symmetry}.

Deep convolutional networks have shown -- see for example \cite{krizhevsky2012imagenet} --
to be the state of the art method for many classification tasks. Several
networks~\cite{ronneberger2015u, long2015fully} leverage huge classification datasets (ImageNet) to provide pixel by
pixel semantic segmentation. They have been successfully used for plant organ
segmentation \cite{shi2019plant}, but are still limited to simple case with a very
small diversity in test datasets. Using simulated data to augment training
datasets is a method which have shown to improve neural network performance in
many fields, e.g~\cite{qiu2016unrealcv, alhaija2018augmented}.

These recent progresses in image processing have made color image processing
much more efficient than 3d point cloud data, for point cloud data is not as
easy to process by convolutional neural networks as structured 2d pictures
(ref.), and 3d voxel volume processing is very computionally expensive (ref. ).

The main bottleneck to uneable the power of convolutional neural network for segmentation
tasks is the need for annotated data. Data annotation is a tedious and time-consuming process
which is a constraint for small scale phenotyping problems. The use of generative
models of plants to feed the machine learning algorithm is a well known method to
reduce the need for hand annotated data.

Plant architecture is a well studied research topic, and many generative model
of plant architecture have been developped in the last decades. Lindenmayer
systems are rewriting systems specifically developped to model plant growth.
They can be used to model arbitrary complex model of plants~\cite{boudon_l-py:_2012}.

In this work, we propose to use plant models to train convolutional neural network for
identification of different plant organs in color images, as well as a
reconstruction of the three dimensional structure of plants from these images.
The specific target application of our method is the identification of organs of the model
plant Arabidopsis thaliana. We describe a way to train neural network on simulated models of
plants in a variable environment rendered with Blender, as well as a simple method for
fine-tuning on real word data. 

The paper is structured as follows: we begin by a first review of the related
litterature. Then, we presend the two and three-dimensional segmentation methods
as well as the simulation environments for virtual plants. A presentation of
quantitative and qualititive results follow. We end the paper with perspectives
and discussions.
