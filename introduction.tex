Plant architecture is a complex geometrical object. Due to both genetic and
environmental variations, organs evolve in patterns which present great diversity
both between species and among individuals of the same species. Automated
reconstruction of plant structure from lab or in-field acquisitions remains a challenge in computer
vision (ref). The most common method for complex measurements of plant structure
remains handmade measurements and image analysis is a major bottleneck in plant phenotyping
\cite{minervini2015image}. The application of automated processing of plant structure are many: precise
quantifying of biomass and yield for agricultural crops, estimation of environmental response
of crops \cite{peirone2018assessing}, mapping genotypes to
phenotypes, estimating growth parameters of species, among other. (Find many
refs)

Deep convolutional networks have shown -- see for example \cite{krizhevsky2012imagenet} --
to be the state of the art method for many classification tasks. Several
networks (\cite{ronneberger2015u}, \cite{long2015fully}) leverage huge classification datasets (ImageNet) to provide pixel by
pixel semantic segmentation. They have been successfully used for plant organ
segmentation \cite{shi2019plant}, but are still limited to simple case with a very
small diversity in test datasets. Using simulated data to augment training
datasets is a method which have shown to improve neural network performance in
many fields, e.g~\cite{qiu2016unrealcv}, \cite{alhaija2018augmented}.

Plant architecture is a well studied research topic, and many generative model
of plant architecture have been developped in the last decades. Lindenmayer
systems are rewriting systems specifically developped to model plant growth.
They can be used to model arbitrary complex model of plants~\cite{boudon_l-py:_2012}.

In this work, we propose to use plant models to train convolutional neural network for
identification of different plant organs. The specific target application of
our method is the identification of organs of the model plant Arabidopsis
thaliana. We describe a data augmentation model using plant models and HDRI
pictures to produce ground truth images, as well as a simple method for
fine-tuning on real word data. We then present qualitative results in various
acquisition condition to assert the robustness of our method.
